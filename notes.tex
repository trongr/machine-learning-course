\documentclass[12pt]{article}
\usepackage{geometry}                % See geometry.pdf to learn the layout options. There are lots.
\geometry{letterpaper}                   % ... or a4paper or a5paper or ...
%\geometry{landscape}                % Activate for for rotated page geometry
%\usepackage[parfill]{parskip}    % Activate to begin paragraphs with an empty line rather than an indent
\usepackage{graphicx}
\usepackage{amsmath,amssymb,amsfonts,amsthm}
\usepackage{epstopdf}

% Computer Concrete
%\usepackage{concmath}
%\usepackage[T1]{fontenc}

% Times variants
%
%\usepackage{mathptmx}
%\usepackage[T1]{fontenc}
%
%\usepackage[T1]{fontenc}
%\usepackage{stix}
%
% Needs to typeset using LuaLaTeX:
%\usepackage{unicode-math}
%\setmainfont{XITS}
%\setmathfont{XITS Math}

\DeclareGraphicsRule{.tif}{png}{.png}{`convert #1 `dirname #1`/`basename #1 .tif`.png}

\theoremstyle{plain}
\newtheorem{thm}{Theorem}
\newtheorem{cor}[thm]{Corollary}
\newtheorem{lem}[thm]{Lemma}
\newtheorem{prop}[thm]{Proposition}
\newtheorem{conj}[thm]{Conjecture}
\newtheorem{quest}[thm]{Question}

\theoremstyle{definition}
\newtheorem{defn}[thm]{Definition}
\newtheorem{e.g.}[thm]{Example}
\newtheorem*{keywords}{Keywords}

\theoremstyle{remark}
\newtheorem{rem}[thm]{Remark}
\newtheorem{note}[thm]{Note}

\title{Coursera Machine Learning Notes}
\author{N. Trong}
\date{\today}                                           % Activate to display a given date or no date

\begin{document}
\maketitle

\begin{note}
Refer to assignment PDF's.
\end{note}

\section{Ex 8. Anomaly Detection and Recommender Systems}

\subsection{Collaborative Filtering Learning Algorithm}

Let $n_m$ be the number of movies, $n_u$ be the number of users. Given rating matrix $Y$ and a number $n$, we want to find a feature matrix $X$ of size $n_m \times n$ and parameter matrix $\Theta$ of size $n_u \times n$, where the $i$-th row of $X$ represents the feature vector for the $i$-th movie, and the $j$-th row of $\Theta$ represents the parameter vector for the $j$-th user. In this context, $n$ represents the number of hidden dimensions of a movie, e.g. $x^i_k$ could refer to say how much action movie $i$ has, $x^i_l$ could refer to how much romance it has, and so on. Similarly, $\theta^j_k$ would refer to how much user $j$ likes action, $\theta^j_l$ how much they like romance.

\begin{note}
These are only example names for the features, since in fact we don't know what features the algorithm will pick up given rating matrix $Y$. The features learned might have nothing to do with common movie genres, for example.
\end{note}

\begin{quest}
Can we cross validate to choose the best value $n$ for the number of hidden features?
\end{quest}

\begin{keywords}
TODO
\end{keywords}

\end{document}
